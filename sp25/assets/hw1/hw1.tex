\documentclass[11pt]{article}
\usepackage{amsmath,amsfonts,amssymb}
\usepackage{hyperref}
\usepackage[margin=1in]{geometry}
\setlength{\parindent}{0pc}
\setlength{\parskip}{10pt}

\title{STAT 165/265 HW 1}
\date{Due Jan 28, 2025}

\begin{document}

\maketitle

\hfill \textbf{Submit to Gradescope by Tuesday, January 28 at 11:59pm.\\Please note that if you do not submit this homework (absent extenuating circumstances that you tell us about in advance) you will be dropped from the course.}

\section*{Logistics Setup}

\emph{Expected completion time: 10 minutes}

Install a time-tracking app as described in the ``Predictions'' section, and submit a screenshot or photo showing it has been 
installed.

\section*{Deliberate Practice}

\emph{Expected completion time: 60 minutes}

Go to \url{https://www.openphilanthropy.org/calibration} and sign in (either create an account or use 
Gmail/Facebook). Click ``Home'' and complete the following exercises in the app:

\begin{itemize}

\item Confidence intervals (level 80\%) -- at least 20 questions
\item 20 questions each from two of the other categories (i.e., two out of PolitiFact, Correlations, City Populations, Math, and Trivia).

\end{itemize}

(Of course, feel free to answer more questions if you would like!)

At the end, go to the ``Results'' tab and take screenshots showing that 
you completed at least 20 questions in each category, as well as your calibration 
performance (the chart that appears below the results).

Write 1-2 sentences about what you learned from this exercise. For which categories were you most/least successful in being calibrated? Were your confidence intervals generally too big or too small?

\section*{Predictions}

\emph{Expected completion time: 35 minutes}

% Make and submit predictions to the questions on this \href{https://forms.gle/ehHmYBWCRxv6bUKw6}{Google Form}. Be sure to follow the format described on the front page of the form.

Register the following predictions. You can submit them by going to 
\url{https://forms.gle/ehHmYBWCRxv6bUKw6} and by following the form's instructions. Be sure to follow the format described on the front page of the form.

\begin{enumerate}
\item[0.] Pick a website, application, or software of your choice. Predict how much time (in minutes) you will spend on it between 12:00am Wednesday January 24th and 11:59pm Sunday January 28th. You should track this {\bf using a time-tracking app} such as \href{https://support.apple.com/en-us/HT210387}{Screen Time} (if you have macOS Catalina or later), \href{https://www.rescuetime.com/}{Rescue Time}, Digital Wellbeing (Android), or any another tracker you prefer.\footnote{Note that some desktop apps record screen time even when an app is open in the background; you might want to be careful about this when making predictions or choosing the application.}

\item[1.] How many people will be in lecture at 1:15pm on Wednesday, January 29th?

      This will resolve based on a count conducted by the course staff.

\item[2.] How many dollars will the movie ``Mufasa: The Lion King'' gross in the US on the weekend of Jan 31--Feb 2?

      This will resolve based on Box Office Mojo's number for Jan 31--Feb 2 \href{https://www.boxofficemojo.com/release/rl615482113/weekend/?ref_=bo_rl_tab#tabs}{here}. 

\item[3.] Will ``Mufasa: The Lion King'' be the highest-grossing movie in the US on the weekend of Jan 31--Feb 2? 

      This will resolve based on this \href{https://www.boxofficemojo.com/weekend/?ref_=bo_nb_wey_secondarytab}{Box Office Mojo page}. Specifically, the question will resolve positively if the entry with row ``Jan 31--Feb 2'' and column ``\#1 Release'' is ``Mufasa: The Lion King,'' and negatively if that entry is another movie.

\end{enumerate}
 
For each question, you will submit a mean and inclusive 80\% confidence interval (or a probability for question 3) 
as well as an explanation of your reasoning (1-2 paragraphs).

For questions 1-3, your prediction (but not the explanation) will appear on the public leaderboard. 
Question 0 will remain private and not count towards the leaderboard.

\end{document}
